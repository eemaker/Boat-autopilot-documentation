%Her beskrives analysearbejdet, dvs. de overvejelser om mulige løsninger I har haft, de løsninger I har valgt og begrundelsen herfor. De grundlæggende valg af hardware- og softwaremæssige komponenter, som er kritiske for realisering af systemet, beskrives her. For at træffe et valg kan der analyseres og diskuteres forskellige løsninger mht. til ydeevne, pris, leveringstid og forhåndskendskab. Disse kan med fordel opstilles i tabelform. Herefter kan der træffes valg, som danner et ”proof of concept” for de forskellige dele af projektet. Der skal selvfølgelig argumenteres for disse valg. Der kan med fordel refereres til detaljer i dokumentet ”Analyse” i projektets bilag.
\newpage
\chapter{Analysis}

From the introduction, several components have to be chosen from a wide variety of viable components, both software and hardware.

The UI is a website made using Bootstrap\cite{bootstrap} and AngularJS\cite{angular}. These were chosen because they are popular and intuitive libraries for implementing websites, and has a lot to offer in terms of design, styling and ease of use. An alternative was React, but the group felt like this would require more work, and was less documented than the others. 

Setting up the map was done with Leaflet, or rather a version of Leaflet called Leaflet-directive\cite{leaflet}. There were many map options, but since it was required that the system could work offline, this eliminated several candidates. With Leaflet-directive, the markers and lines were easy to set up and change, and the library had some built-in control schemes that lived up to the expectations of the group.

The website hosted on a NodeJS server\cite{nodejs}, this was chosen because NodeJS is incredibly easy to set up and host locally for testing purposes, and requires very few packages installed to get working. These and other factors make it a very fast and responsive server solution. It is also written in JavaScript, which was a good fit when the group was working with JS on the client side as well.

The controller is a Raspberry Pi 3b, and this was chosen for a variety of reasons. The RPi 3b was one among 16 candidates that were considered for the system controller. A detailed analysis of this can be found in the documentation in section~\ref{sec:hardware_design} on page~\pageref{sec:hardware_design}, but ultimately the choice came down primarily to price and available documentation, and the latter would turn out to be relevant later on.

The GPS is a Ublox Neo 7-M. This unit offers reasonable performance for a non-RTK GPS, which is the only reasonable type to use in the system. Only bigger projects with a lot more funding would invest in an RTK GPS, whose price can easily exceed tens of thousands of DKK. The availability of this unit in the workshop contributed to its selection, but other similar GPS modules didn't offer considerably different performance, so the choice ultimately fell on the Ublox.

GeographicLib was used for geographic calculations, specifically for calculating rhumb lines and distances. Early in the development of the navigation unit, these functions were programmed manually, but it turned out that the algorithms offered by GeographicLib were much faster and more precise than our own. It was then decided to use GeographicLib, but exclusively for these two functions.

The MiniPID \cite{minipid} library was used to implement the Autopilot PID loops, since it is widely used when implementing PIDs in microcontrollers.

PiGpio \cite{pigpio} was chosen for controlling the Raspberry Pi pin outputs. Initially it seemed like using Python scripts would be the way to control the outputs, but this C++ library was recommend by many developers, and contained the required functionality for the motor classes to access the corresponding hardware.

The NMEA\cite{NMEA} protocol was used for communication between the GPS, Navigation, and Autopilot components, as specified in the original problem description given by EIVA. Several write-ups of the original NMEA-0183 protocol were found, and the necessary command message structures were identified and implemented.

A C++ JSON library, created by Niels Lohmann\cite{json}, was used to parse to and from the JSON format used in shared data files.

Boost test and FakeIt were chosen as the test frameworks, for more information on this, see the Test section of this document.

%TODO vi burger boost

