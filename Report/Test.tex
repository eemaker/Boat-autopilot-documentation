%Her beskrives test af systemet, herunder en overordnet beskrivelse af hvordan modultest, integrationstest og accepttest er udført. For detaljer angives reference til jeres ”testdokument” i projektets bilag
\chapter{Test}
\label{sec:test}
\section{Software tests}
Testing was not initially something both team members had long experience with, but it soon became clear that verification would become an important part of the development process. 

One group member had taken software testing the previous semester, and presented the general ideas and concepts before any controller code was written. 

The team decided that unit testing would be mandatory for all C++ code that was possible to test, since it offered great benefits, and potentially a whole new way of thinking about software development. This also means that all classes that requires testing need an interface, to be able to make mock and fake classes.

After debating and attempting different testing frameworks, it was decided that the Boost unit testing framework and the FakeIt isolation framework would be used. 

This combination proved to be very effective, but since both members of the team had no experience with either framework, there were a few issues, and a lot of learning, before the frameworks were used to their full potential.

At the end of development, over 170 tests totaling more than 8000 lines of code were written. 

\subsection{Unit tests}

Listing ~\ref{lst:servo_test} shows an example of a unit test of the Servo class.

\begin{lstlisting}[caption = {Servo test of SetPosition}, captionpos=b, label={lst:servo_test}, language=C++,firstnumber=1]
BOOST_AUTO_TEST_CASE(Servo_test_SetPosition_50_returns_1500)
{
	fakeit::Mock<IGPIO> gpioMock;
	fakeit::Fake(Method(gpioMock, GpioServo));
	fakeit::Fake(Method(gpioMock, GpioSetMode));

	IGPIO & gpio = gpioMock.get();

	Servo uut(gpio);

	uut.SetPosition(50);

	fakeit::Verify(Method(gpioMock, GpioServo).Matching([](auto gpio, auto pulsewidth)
	{
		//Ignore the gpio, since it could change and is not important
		return pulsewidth == 1500;
	})).Exactly(2); //It should call it once in the constructor and once again when we set it.
}
\end{lstlisting}

Unit testing has three phases, arrange, act, and assert. 

In the example, mocks and fakes are first set up, or arranged, to isolate the test from its external dependencies; other classes and functions. This way, an error in a function will only lead to errors when testing that function, and not when testing other functions which depend on it. 

The unit under test (uut) is then constructed, and a function is called; this is the act step. 

Finally, the assert step verifies that the function did exactly what was intended, in this case calling another function with a specific argument matching a specific value. This is done using a lambda function.

Most unit tests in the project are much more complicated than this simple example, but the general structure and procedure is the same. 

\subsection{Integration tests}

For some of the integration tests, the previous unit tests were repeated without the mocks and fakes, to verify that everything still behaves as desired when the real classes are constructed instead of fakes and mocks from the interfaces.

During this phase, a few design errors were found, including a dependency problem between the transmitter and navigation units. Without the unit test base, it is doubtful that this and other issues would have been resolved as quickly as they did, since they required some restructuring of those classes. 

Finally, two special tests were written, verifying the entire code base including the user interface. In these tests, the GPS inputs and motor outputs were mocked, but the rest of the classes and other elements were constructed. The user interface was then started, and commands sent through the .json files. The first test calculates and traverses a point to point path, and shows the progress and finish in the user interface. The second test calculates a coverage path and shows it in the user interface.

These two non-automatic tests were useful when verifying the final product, and enabled the development team to easily build a main program to run the acceptance test, since the entire code base's functionality had been verified.

The integration tests were thus instrumental in sanitizing the code base and the dependency trees between classes.

\section{Hardware tests}
As opposed to software testing the group has had a lot of experience testing hardware. All of the components in the system are off-the-shelf, therefore it can be assumed that their design has been tested elsewhere. This means that the hardware testing for this project is focused around the interfaces of the components.

The step-down regulators were tested by connecting them to a bench power supply, and the outputs were connected to a voltmeter.

The motor controller was tested by supplying it with the voltage it needed, and driving its PWM input with a frequency generator, then measuring its output with an oscilloscope. 

The GPS receiver was tested by connecting it to a sample application by u-blox, which displays the messages the GPS receiver is outputting in NMEA format.

Similar test were conducted for all components in the system. 

To integrate all of the components, most of the leg work is done, since all of the interfaces have been tested. The integration test is then a matter of connecting the components one by one and testing the combined interface of the components. To test the system with the Raspberry Pi, some test code had to be written, to get the PWM outputs and read the GPS receiver's signals.


\section{Acceptance test}
The acceptance test was conducted outside, to ensure a GPS connection. The boat was "piloted" by one of the team members (by carrying it), while the other sent commands using the user interface on the website. The reason for doing the acceptance test out of water was because of the risk of capsizing, which could result in shorting of electronics. It is also possible that the boat would sail to a position out of reach, resulting in the boat being lost.

The boat is set up in such a way that it connects to a WiFi network called WindowsAP with a specific password. The website is then hosted on the host name "http://\-raspberrypi.\-local". Between all of the test the website was refreshed to ensure a clean slate for the next test.