%Der skal udarbejdes et resumé på både på dansk og på engelsk (abstract). De skal begge være en kortfattet sammenfatning af projektrapporten og bør indeholde følgende: En kort oversigt over den faglige gennemgang og konklusionen. Det bør indledes med en meget kort redegørelse for emnet, afgrænsningen og synsvinklen. Formålet er kun at fortælle læseren, hvad denne projektrapport kan bruges til. Resuméet skal kunne læses selvstændigt og må ikke indeholde henvisninger til afsnit i rapporten. Skriv så konkret som muligt, og undgå vage udtryk, fyldord, subjektive beskrivelser etc. Resuméets omfang er i størrelsesordenen 1/3 side pr. sprog
\selectlanguage{danish}
\begin{abstract}
	Denne rapport beskriver udviklingsforløbet af en selv-sejlende båd der kan navigere havområder, der muliggør kortlægning af havområder, kaldet CAPTAIN. Flere meget forskellige moduler skulle udvikles til systemet, og skulle fungere samtidigt for at opfylde kravene.
	
	Systemet har en brugergrænseflade på en hjemmeside med et interaktiv kort der giver brugeren mulighed for at give kommandoer til at sejle til et bestemt sted, eller at kortlægge et bestemt område. Kommandoerne sendes til en server som sender disse videre til en kontrol-enhed.
	
	Kontrol-enheden anvender navigations-algoritmer til at beregne og følge en sejlrute på autopilot med hjælp fra en GPS-modtager, og give feedback til brugeren om hvor langt man er nået. 
	
	Brugeren kan også se diagnostiske øjebliks-data for båden, og endda ændre nogle af parametrene der bruges i navigationsalgoritmerne og autopiloten.
	
	Resultatet af projektet er en fuldt ud funktionsdygtig prototype der klarede alle tests og opfylder formålet.
\end{abstract}