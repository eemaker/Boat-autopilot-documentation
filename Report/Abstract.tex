%Der skal udarbejdes et resumé på både på dansk og på engelsk (abstract). De skal begge være en kortfattet sammenfatning af projektrapporten og bør indeholde følgende: En kort oversigt over den faglige gennemgang og konklusionen. Det bør indledes med en meget kort redegørelse for emnet, afgrænsningen og synsvinklen. Formålet er kun at fortælle læseren, hvad denne projektrapport kan bruges til. Resuméet skal kunne læses selvstændigt og må ikke indeholde henvisninger til afsnit i rapporten. Skriv så konkret som muligt, og undgå vage udtryk, fyldord, subjektive beskrivelser etc. Resuméets omfang er i størrelsesordenen 1/3 side pr. sprog
\selectlanguage{english}
\begin{abstract}
	This project details the development of an autonomous naval surveying vessel in the fall of 2017 named CAPTAIN. Several very different modules had to be developed for the system and work together simultaneously to fulfill the requirements.
	
	The system has a website user interface with an interactive map to enable the user to give commands to sail to a certain coordinate or cover an area. The commands are sent to a server which passses them on to a controller unit. 
	
	The controller can use its navigation algorithms to calculate and traverse paths on autopilot with the help of a GPS receiver, and give feedback to the user about progress.
	
	A technician is able to view live diagnostics data, and even to modify some of the parameters used within the navigation algorithms and the autopilot in real-time.
	
	The results of this project was a fully functioning prototype that was extensively tested and verified.
\end{abstract}