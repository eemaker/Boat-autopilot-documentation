%Her beskrives hvad der mangler for at gøre projektet færdigt, og hvilke fremtidige udvidelses- og anvendelsesmuligheder, der er i projektet.
\newpage
\chapter{Future work}
There are a lot of things that could be bettered in the CAPTAIN project, and there are also some additions that would just make for a better product.

One thing that was labeled under Won't in the MoSCoW is polygon coverage, this was simply to do with the fact that polygon coverage is a very large subject. Usually when one wants to cover an area it is not a square. Even then, generally it's not square with latitudinal lines, like the system is limited to at this time. 

Another thing that would be a nice to have is a way to navigate a non-straight point to point path. If the user for example wants to cover a river, this would be very impractical with the current point to point implementation. 

The idea of saving an already calculated path, so it can be run in the exact same way at a later time would make it possible to compare data collected on separate occasions and eliminate the positional differences as a factor. This functionality would have been implemented, had there been time enough to develop it, and some preliminary work was done to enable this, but later scrapped due to lack of time.

A thing that might not be optimal in the system at the moment is the communication between the website and the controller. It is handled by reading and writing to files, which can be slow and potentially unsafe. This was done for simplicity, but it might be a better idea to make the controller host a http server so it can handle GET and POST requests itself instead of relying on the files.  

With the current implementation, the controller only supports one boat type. But the software design is built from rather generic interfaces so it should be possible in the future to include a variety of boat types.

One thing that we were intending to get done for the project hand-in was anecho sounding device, or one simulated in software, that would measure the depth of the sea bed below the boat. We didn't, however, have time to implement this feature.

The system could be classified as a mobile autonomous robot, but at this stage it does not have obstacle avoidance. This could be used to make sure that the boat is not going to run ashore, hit a bouy or even another naval vessel. 

If the system was ever to become anything other then a scale test bed, then it would both need a internet connection via GSM and probably an RTK GPS receiver. A real time kinematic GPS receiver features sub-centimeter precision and some can have interpolated position updates every 1/10 of a second, which could make the autopilot much more stable.

Lastly on the topic of sensors, it would probably be a good idea to have some sort of battery monitoring system, if the system is going to be powered by battery. 

% Polygon coverage
% Ikke lige linke p2p
% At kunne gemme paths
% Sensor integration
% Manuel styring af båden
% Bedre kommunikations protokol. ei ikke filer men post call direkte til c++ programmet.
% Andre båd typer
% obstacle avoidance
% rtk gps
% Gsm
% Battery måler