%Her beskrives hvordan I udførte arbejdsfordelingen i fm. produktets udvikling. Valgte I den sikre vej, hvor I hver kunne anvende jeres forhåndsviden – eller valgte I en udfordrende vej, hvor I hver bevidst tilstræbte at arbejde med ukendte områder? Hvis I har anvendt Scrum kan anvendelsen af Scrum taskboard beskrives her. I kan her beskrive kort hvordan arbejdsfordelingen fungerede.


\chapter{Work distribution}
For managing the project we used a SCRUM-like task board. First off the task board was on a physical bulletin board, but later it was moved to the online tool Trello \cite{trello}. 

The task board was divided up into five sections; Backlog, Doing, Review, Reviewed, and Done. The backlog is the place where all the tasks for the project reside, that is every thing from implementation task, to documentation or research. The Doing section is used to signify that a task is currently being worked on, this minimizes the risk of parallel work. When a task in doing is done it is move to the Review section, where the other group member looks at what has been done and gives comments. After review the task is moved to the Reviewed section, where the original owner of the task decides whether it goes back in the backlog for further work, or it goes to the Done section.

Even though the task board is a SCRUM-like board, we did not follow normal SCRUM, see the Development process section.

The distribution of tasks was rather simple; if a team member found a task interesting and didn't have any exisiting critical tasks, they could take it. Many tasks had prerequisite tasks which needed to be completed before the task could be started. That way, the less interesting task also gets done in a timely fashion. In the beginning of the project we also used the task board, but a lot of the tasks were done together in this early stage. 

This way of distributing the work meant that both team members were able to take tasks they didn't have familiarity with, and learn from the experience. This way of handling task distribution is rather simple, but it enabled the small team to work fast and learn from the process.