\chapter{Acceptance test}
With some well defined use cases, an acceptance test can be constructed. It is important that there is a test for all the main scenarios and all exceptions and extension. 

It important to realize that this chapter does not contain the results for the acceptance tests, but only the actions and exceptions. The results can be found in chapter~\ref{sec:acceptance_test_results} on page~\pageref{sec:acceptance_test_results}.

\section{Test equipment}
To complete these test a tester is going to need a computer with WiFi or Ethernet. A outdoor pool of water is all so necessary, since the GPS need view of the satellites, and the boat needs some water to move in. 

\section{Tests}
A test is built up of one or more steps that a tester has to perform, and then see if the expected result match what they see.
As the tests are built from the use cases they inherit the same prerequisites.

A couple of things that a tester might need to know; default parameters are some parameters that get set when the system don't know what else to put, the specific values are defined by the implementer, but might be something like 1 or 0 or even -1. When the ''edit section of the user interface'' is mentioned here is referred to the section of the edit parameters menu, where in the tester can change the given parameters. This is seen on the user interface mockup on figure~\ref{fig:mockup:edit_param}, where it says \texttt{Edit :"Profile 1"}.


\begin{table}[H] 			
	\centering
	\begin{tabularx}{\textwidth}{|c|X|X|}
		\hline
		\bfseries Use case under test & \multicolumn{2}{|c|}{ Use case 1 - New parameter profile} \\ \hline
		\bfseries Scenario & \multicolumn{2}{| c |}{Main scenario} \\ \hline
		\bfseries Prerequisite &  \multicolumn{2}{|l|}{\asg{\shortstack[l]{
				\textbullet"Edit parameter" menu is open in user interface
			}}} \\  \hline
		\bfseries Step  & \bfseries Action &  \bfseries Expected \\ \hline 
		%Copy below when adding more steps
		1 & Press the "New Profile" button on the user interface & A new parameter profile should appear in the list of parameter profiles, and the new parameter profile should be loaded i the edit section of the user interface. \\ \hline
	\end{tabularx}
	\caption{Test of: Use case 1 - New parameter profile - Main scenario}
\end{table}

\begin{table}[H] 			
	\centering
	\begin{tabularx}{\textwidth}{|c|X|X|}
		\hline
		\bfseries Use case under test & \multicolumn{2}{|c|}{ Use case 2 - Delete parameter profile} \\ \hline
		\bfseries Scenario & \multicolumn{2}{| c |}{Main Scenario} \\ \hline
		\bfseries Prerequisite &  \multicolumn{2}{|l|}{\asg{\shortstack[l]{
				\textbullet "Edit parameter" menu is opened in user interface\\
				\textbullet A parameter profile must exist
			}}} \\  \hline
		\bfseries Step  & \bfseries Action &  \bfseries Expected \\ \hline 
		%Copy below when adding more steps
		1 & Select a parameter profile from the list of parameter profiles & The parameter profile selected is highlighted\\ \hline
		2 & Press the "Delete profile" button in the user interface & The previously selected parameter profile should be delete from the list of parameter profiles\\ \hline
	\end{tabularx}
	\caption{Test of: Use case 2 - Delete parameter profile - Main scenario}
\end{table}

\begin{table}[H] 			
	\centering
	\begin{tabularx}{\textwidth}{|c|X|X|}
		\hline
		\bfseries Use case under test & \multicolumn{2}{|c|}{ Use case 2 - Delete Parameter profile} \\ \hline
		\bfseries Scenario & \multicolumn{2}{| c |}{Extension 1: Active profile deleted} \\ \hline
		\bfseries Prerequisite &  \multicolumn{2}{|l|}{\asg{\shortstack[l]{
				\textbullet "Edit parameter" menu is opened in user interface\\
				\textbullet A parameter profile must exist
		}}} \\  \hline
		\bfseries Step  & \bfseries Action &  \bfseries Expected \\ \hline 
		%Copy below when adding more steps
		1 & Select a parameter profile from the list of parameter profiles & The parameter profile selected is highlighted\\ \hline
		2 & Press the "Set Profile" button in the user interface  & The name under the text "Active Profile:" now corresponds with previously selected profile\\ \hline
		3 & Make sure that the now active profile is still selected, then press the "Delete profile" button in the user interface & The previously selected parameter profile should be deleted from the list of parameter profiles, and under the text "Active Profile" it should now say "None" \\ \hline
	\end{tabularx}
	\caption{Test of: Use case 2 - Delete parameter profile - Extension 1: Active profile deleted}
\end{table}

\begin{table}[H] 			
	\centering
	\begin{tabularx}{\textwidth}{|c|X|X|}
		\hline
		\bfseries Use case under test & \multicolumn{2}{|c|}{ Use case 3 - Edit parameter profile} \\ \hline
		\bfseries Scenario & \multicolumn{2}{| c |}{Main  scenario} \\ \hline
		\bfseries Prerequisite &  \multicolumn{2}{|l|}{\asg{\shortstack[l]{
					\textbullet "Edit parameter" menu is opened in user interface\\
					\textbullet A Parameter profile must exist
		}}}\\  \hline
		\bfseries Step  & \bfseries Action &  \bfseries Expected \\ \hline 
		%Copy below when adding more steps
		1 & Select a parameter profile form the list of parameter profiles & The selected parameter profile should be highlighted\\ \hline
		2 & Press the "Edit Profile" button in the user interface & The parameter profile values should be loaded and displayed in the edit section of the user interface\\ \hline
		3 & Change the values either by moving the sliders or by directly entering values in the input fields & The values changed should correspond to what they were changed to\\ \hline
		4 & Press the "Save" button & Nothing should visually happen, but the values should be saved.\\ \hline
	\end{tabularx}
	\caption{Test of: Use case 3 - Edit parameter profile - Main scenario}
\end{table}

\begin{table}[H] 			
	\centering
	\begin{tabularx}{\textwidth}{|c|X|X|}
		\hline
		\bfseries Use case under test & \multicolumn{2}{|c|}{ Use case 3 - Edit parameter profile} \\ \hline
		\bfseries Scenario & \multicolumn{2}{| c |}{Extension 1: Revert} \\ \hline
		\bfseries Prerequisite &  \multicolumn{2}{|l|}{\asg{\shortstack[l]{
					\textbullet "Edit parameter" menu is opened in user interface\\
					\textbullet A Parameter profile must exist
		}}}\\  \hline
		\bfseries Step  & \bfseries Action &  \bfseries Expected \\ \hline 
		%Copy below when adding more steps
		1 & Select a parameter profile form the list of parameter profiles & The selected parameter profile should be highlighted\\ \hline
		2 & Press the "Edit Profile" button in the user interface & The parameter profile values should be loaded and displayed in the edit section of the user interface\\ \hline
		3 & Change the values either by moving the sliders or by directly entering values in the input fields & The values changed should correspond to what they were changed to\\ \hline
		4 & Press the "Revert" button & The values changed should now go back to what they were set to before they were changed\\ \hline
	\end{tabularx}
	\caption{Test of: Use case 3 - Edit parameter profile - Extension 1: Revert}
\end{table}

\begin{table}[H] 			
	\centering
	\begin{tabularx}{\textwidth}{|c|X|X|}
		\hline
		\bfseries Use case under test & \multicolumn{2}{|c|}{ Use case 4 - Set active parameter profile} \\ \hline
		\bfseries Scenario & \multicolumn{2}{| c |}{Main scenario} \\ \hline
		\bfseries Prerequisite &  \multicolumn{2}{|l|}{\asg{\shortstack[l]{
					\textbullet "Edit parameter" menu is open in user interface\\
					\textbullet A parameter profile must exist
		}}}\\  \hline
		\bfseries Step  & \bfseries Action &  \bfseries Expected \\ \hline 
		%Copy below when adding more steps
		1 & Select a parameter profile form the list of parameter profiles & The selected parameter profile should be highlighted\\ \hline
		2 & Press the "Set Profile" button in the user interface  & The name under the text "Active Profile:" now corresponds with previously selected profile\\ \hline
	\end{tabularx}
	\caption{Test of: Use case 4 - Set active parameter profile}
\end{table}

%TODO Use case 4, consider if bullet point number 5 makes sense

%TODO Concider if there are missing prerequsites in use case 5
\begin{table}[H] 			
	\centering
	\begin{tabularx}{\textwidth}{|c|X|X|}
		\hline
		\bfseries Use case under test & \multicolumn{2}{|c|}{ Use case 5 - Request diagnostics} \\ \hline
		\bfseries Scenario & \multicolumn{2}{| c |}{Main scenario} \\ \hline
		\bfseries Prerequisite &  \multicolumn{2}{|l|}{\asg{\shortstack[l]{
					None
		}}}\\  \hline
		\bfseries Step  & \bfseries Action &  \bfseries Expected \\ \hline 
		%Copy below when adding more steps
		1 & Select the "Diagnostics" menu in the user interface & The Diagnostics menu should open, and it should display diagnostics data for the gps, motor and connection to the boat, as well as the boats current position and orientation\\ \hline
	\end{tabularx}
	\caption{Test of: Use case 5 - Request Diagnostics - Main scenario}
\end{table}

\begin{table}[H] 			
	\centering
	\begin{tabularx}{\textwidth}{|c|X|X|}
		\hline
		\bfseries Use case under test & \multicolumn{2}{|c|}{ Use case 6 - Set point to point destination} \\ \hline
		\bfseries Scenario & \multicolumn{2}{| c |}{Main scenario} \\ \hline
		\bfseries Prerequisite &  \multicolumn{2}{|l|}{\asg{\shortstack[l]{
					\textbullet The user interface has an updated map\\
					\textbullet The "Point to point" menu is opened in user interface
		}}}\\  \hline
		\bfseries Step  & \bfseries Action &  \bfseries Expected \\ \hline 
		
		1 & Select a position on the map & The user interface should indicate where you have pressed on map, and it should update the "latitude" and "longitude" input fields to correspond with latitude and longitude of the selected position on the map\\ \hline
	\end{tabularx}
	\caption{Test of: Use case 6 - Set point to point destination - Main scenario}
\end{table}

\begin{table}[H] 			
	\centering
	\begin{tabularx}{\textwidth}{|c|X|X|}
		\hline
		\bfseries Use case under test & \multicolumn{2}{|c|}{ Use case 6 - Set point to point destination} \\ \hline
		\bfseries Scenario & \multicolumn{2}{| c |}{Alternate flow 1: Coordinate input} \\ \hline
		\bfseries Prerequisite &  \multicolumn{2}{|l|}{\asg{\shortstack[l]{
					\textbullet The user interface has an updated map\\
					\textbullet The "Point to point" menu is opened in user interface
		}}}\\  \hline
		\bfseries Step  & \bfseries Action &  \bfseries Expected \\ \hline 
		%Copy below when adding more steps
		1 & Input a coordinate by writing it in the input fields "longitude" and "latitude & On the map there should be an indication of the coordinate inputed in the input fields\\ \hline
	\end{tabularx}
	\caption{Test of: Use case 6 - Set point to point destination - Alternate flow 1: Coordinate input}
\end{table}


\begin{table}[H] 			
	\centering
	\begin{tabularx}{\textwidth}{|c|X|X|}
		\hline
		\bfseries Use case under test & \multicolumn{2}{|c|}{ Use case 7 - Calculate point to point path} \\ \hline
		\bfseries Scenario & \multicolumn{2}{| c |}{Main scenario} \\ \hline
		\bfseries Prerequisite &  \multicolumn{2}{|l|}{\asg{\shortstack[l]{
					\textbullet Use case 6 - Set point to point destination has been completed\\
					\textbullet The user interface has an updated map\\
					\textbullet The "Point to point" menu is opened in user interface\\
					\textbullet The boat must not be running a navigation task
		}}}\\  \hline
		\bfseries Step  & \bfseries Action &  \bfseries Expected \\ \hline 
		%Copy below when adding more steps
		1 & Select a position on the map & The user interface should indicate where you have pressed on map, and it should update the "latitude" and "longitude" input fields to correspond with latitude and longitude of the selected position on the map\\ \hline
		2 & Press the "Calculate path" button in the user interface & The button should change text to "Calculating..." for a while while it is calculating, After calculating the calculated path should be displayed on the map, from the boat to the selected position on the map. Also after calculating the "Calculating..." button should change to "Start"\\ \hline
	\end{tabularx}
	\caption{Test of: Use case 8 - Calculate point to point path - Main scenario}
\end{table}

\begin{table}[H] 			
	\centering
	\begin{tabularx}{\textwidth}{|c|X|X|}
		\hline
		\bfseries Use case under test & \multicolumn{2}{|c|}{ Use case 8 - Run point to point path} \\ \hline
		\bfseries Scenario & \multicolumn{2}{| c |}{Main scenario} \\ \hline
		\bfseries Prerequisite &  \multicolumn{2}{|l|}{\asg{\shortstack[l]{
					\textbullet The user interface has an updated map\\
					\textbullet The "Point to point" men is opened in user interface\\
					\textbullet There must be an active parameter profile\\
					\textbullet The boat must not be running a navigation task
		}}}\\  \hline
		\bfseries Step  & \bfseries Action &  \bfseries Expected \\ \hline 
		%Copy below when adding more steps
		1 & Select a position on the map & The user interface should indicate where you have pressed on map, and it should update the "latitude" and "longitude" input fields to correspond with latitude and longitude of the selected position on the map\\ \hline
		2 & Press the "Calculate path" button in the user interface & The button should change text to "Calculating..." for a while while it is calculating, After calculating the calculated path should be displayed on the map, from the boat to the selected position on the map. Also after calculating the "Calculating..." button should change to "Start"\\ \hline
		3 & Press the "Start" button in the user interface & The button should change to "Running...", and the boat starts moving and keeps on doing so until it has reached the position chosen in the first step, also while the boat is moving the ETA on the user interface is continually updated with an approximate ETA. On the map the icon of the boat should also change to correspond with the real position of the boat. After the boat is at the destination the "Running..." button should change to "Calculate path"\\ \hline
	\end{tabularx}
	\caption{Test of: Use case 9 - Run point to point path - Main scenario}
\end{table}

\begin{table}[H] 			
	\centering
	\begin{tabularx}{\textwidth}{|c|X|X|}
		\hline
		\bfseries Use case under test & \multicolumn{2}{|c|}{ Use case 9 - Stop point to point path} \\ \hline
		\bfseries Scenario & \multicolumn{2}{| c |}{Main scenario} \\ \hline
		\bfseries Prerequisite &  \multicolumn{2}{|l|}{\asg{\shortstack[l]{
					\textbullet The user interface has an updated map\\
					\textbullet The "Point to point" men is opened in user interface\\
					\textbullet There must be an active parameter profile
		}}}\\  \hline
		\bfseries Step  & \bfseries Action &  \bfseries Expected \\ \hline 
		%Copy below when adding more steps
		1 & Select a position on the map & The user interface should indicate where you have pressed on map, and it should update the "latitude" and "longitude" input fields to correspond with latitude and longitude of the selected position on the map\\ \hline
		2 & Press the "Calculate path" button in the user interface & The button should change text to "Calculating..." for a while while it is calculating, After calculating the calculated path should be displayed on the map, from the boat to the selected position on the map. Also after calculating the "Calculating..." button should change to "Start"\\ \hline
		3 & Press the "Start" button in the user interface & The button should change to "Running...", and the boat starts moving. While the boat is moving the ETA on the user interface is continually updated with an approximate ETA. On the map the icon of the boat should also change to correspond with the real position of the boat.\\ \hline
		4 & Press the "Stop" button in the user interface, before the boat has reached its destination & In the user interface the button "Running..." changes to "Calculate path". The boats turns off its motors and comes to a halt.\\ \hline
	\end{tabularx}
	\caption{Test of: Use case 9 - Stop point to point path - Main scenario}
\end{table}

\begin{table}[H] 			
	\centering
	\begin{tabularx}{\textwidth}{|c|X|X|}
		\hline
		\bfseries Use case under test & \multicolumn{2}{|c|}{ Use case 10 - Set coverage area} \\ \hline
		\bfseries Scenario & \multicolumn{2}{| c |}{Main scenario} \\ \hline
		\bfseries Prerequisite &  \multicolumn{2}{|l|}{\asg{\shortstack[l]{
					\textbullet The user interface has an updated map\\
					\textbullet The "Coverage" menu is opened in user interface
		}}}\\  \hline
		\bfseries Step  & \bfseries Action &  \bfseries Expected \\ \hline 
		%Copy below when adding more steps
		1 & Select a position on the map & The map indicates the position and labels it "1" to indicate the starting position of the coverage, the input fields for the position update correspondingly\\ \hline
		2 & Select a different position on the map & The map indicates the position and labels it "2" to indicate the end position of the coverage, the input fields for the position update correspondingly\\ \hline
	\end{tabularx}
	\caption{Test of: Use case 10 - Set coverage area - Main scenario}
\end{table}

\begin{table}[H] 			
	\centering
	\begin{tabularx}{\textwidth}{|c|X|X|}
		\hline
		\bfseries Use case under test & \multicolumn{2}{|c|}{ Use case 10 - Set coverage area} \\ \hline
		\bfseries Scenario & \multicolumn{2}{| c |}{Alternate flow 1: Coordinate inputs} \\ \hline
		\bfseries Prerequisite &  \multicolumn{2}{|l|}{\asg{\shortstack[l]{
					\textbullet The user interface has an updated map\\
					\textbullet The "Coverage" menu is opened in user interface
		}}}\\  \hline
		\bfseries Step  & \bfseries Action &  \bfseries Expected \\ \hline 
		%Copy below when adding more steps
		1 & Input coordinate for point "1" by writing it in the input fields "longitude" and "latitude" & The map indicates the position and labels it "1" to indicate the starting position of the coverage, the input fields for the position update correspondingly\\ \hline
		2 & Input coordinate for point "2" by writing it in the input fields "longitude" and "latitude" & The map indicates the position and labels it "2" to indicate the end position of the coverage, the input fields for the position update correspondingly\\ \hline
	\end{tabularx}
	\caption{Test of: Use case 10 - Set coverage area - Alternate flow 1: Coordinate inputs}
\end{table}

\begin{table}[H] 			
	\centering
	\begin{tabularx}{\textwidth}{|c|X|X|}
		\hline
		\bfseries Use case under test & \multicolumn{2}{|c|}{ Use case 11 - Calculate coverage path} \\ \hline
		\bfseries Scenario & \multicolumn{2}{| c |}{Main scenario} \\ \hline
		\bfseries Prerequisite &  \multicolumn{2}{|l|}{\asg{\shortstack[l]{
					\textbullet The user interface has an updated map\\
					\textbullet The "Coverage" menu is opened in user interface\\
					\textbullet The boat must not be running a navigation task
		}}}\\  \hline
		\bfseries Step  & \bfseries Action &  \bfseries Expected \\ \hline 
		%Copy below when adding more steps
		1 & Input coordinate for point "1" by writing it in the input fields "longitude" and "latitude" & The map indicates the position and labels it "1" to indicate the starting position of the coverage, the input fields for the position update correspondingly\\ \hline
		2 & Input coordinate for point "2" by writing it in the input fields "longitude" and "latitude" & The map indicates the position and labels it "2" to indicate the end position of the coverage, the input fields for the position update correspondingly\\ \hline
		3 & Press the "Calculate path" button in the user interface & The button should change text to "Calculating..." for a while, while it is calculating, After calculating the calculated path should be displayed on the map, from the boat to the endpoint of the Bézier curves covering the selected area. Also after calculating, the "Calculating..." button should change to "Start"\\ \hline
	\end{tabularx}
	\caption{Test of: Use case 11 - Calculate coverage path - Main scenario}
\end{table}

\begin{table}[H] 			
	\centering
	\begin{tabularx}{\textwidth}{|c|X|X|}
		\hline
		\bfseries Use case under test & \multicolumn{2}{|c|}{ Use case 12 - Run coverage path} \\ \hline
		\bfseries Scenario & \multicolumn{2}{| c |}{Main scenario} \\ \hline
		\bfseries Prerequisite &  \multicolumn{2}{|l|}{\asg{\shortstack[l]{
					\textbullet The user interface has an updated map\\
					\textbullet The "Coverage" menu is opened in user interface\\
					\textbullet There must be an active parameter profile\\
					\textbullet The boat must not be running a navigation task
		}}}\\  \hline
		\bfseries Step  & \bfseries Action &  \bfseries Expected \\ \hline 
		%Copy below when adding more steps
		1 & Input coordinate for point "1" by writing it in the input fields "longitude" and "latitude" & The map indicates the position and labels it "1" to indicate the starting position of the coverage, the input fields for the position update correspondingly\\ \hline
		2 & Input coordinate for point "2" by writing it in the input fields "longitude" and "latitude" & The map indicates the position and labels it "2" to indicate the end position of the coverage, the input fields for the position update correspondingly\\ \hline
		3 & Press the "Calculate path" button in the user interface & The button should change text to "Calculating..." for a while, while it is calculating, After calculating the calculated path should be displayed on the map, from the boat to the endpoint of the Bézier curves covering the selected area. Also after calculating, the "Calculating..." button should change to "Start"\\ \hline
		4 & Press the "Start" button in the user interface & The button should change to "Running...", and the boat starts moving and keeps on doing so until it has completed the coverage of the area chosen in the first step, also while the boat is moving the ETA on the user interface is continually updated with an approximate ETA. On the map the icon of the boat should also change to correspond with the real position of the boat. After the boat is at the endpoint the "Running..." button should change to "Calculate path"\\ \hline
	\end{tabularx}
	\caption{Test of: Use case 12 - Run coverage path - Main scenario}
\end{table}

\begin{table}[H] 			
	\centering
	\begin{tabularx}{\textwidth}{|c|X|X|}
		\hline
		\bfseries Use case under test & \multicolumn{2}{|c|}{ Use case 13 - Stop coverage path} \\ \hline
		\bfseries Scenario & \multicolumn{2}{| c |}{Main scenario} \\ \hline
		\bfseries Prerequisite &  \multicolumn{2}{|l|}{\asg{\shortstack[l]{
					\textbullet The user interface has an updated map\\
					\textbullet The "Coverage" men is opened in user interface\\
					\textbullet There must be an active parameter profile
		}}}\\  \hline
		\bfseries Step  & \bfseries Action &  \bfseries Expected \\ \hline 
		%Copy below when adding more steps
		1 & Input coordinate for point "1" by writing it in the input fields "longitude" and "latitude" & The map indicates the position and labels it "1" to indicate the starting position of the coverage, the input fields for the position update correspondingly\\ \hline
		2 & Input coordinate for point "2" by writing it in the input fields "longitude" and "latitude" & The map indicates the position and labels it "2" to indicate the end position of the coverage, the input fields for the position update correspondingly\\ \hline
		3 & Press the "Calculate path" button in the user interface & The button should change text to "Calculating..." for a while, while it is calculating, After calculating the calculated path should be displayed on the map, from the boat to the endpoint of the Bézier curves covering the selected area. Also after calculating, the "Calculating..." button should change to "Start"\\ \hline
		4 & Press the "Start" button in the user interface & The button should change to "Running...", and the boat starts moving, also while the boat is moving the ETA on the user interface is continually updated with an approximate ETA. On the map the icon of the boat should also change to correspond with the real position of the boat. \\ \hline
		5 & Press the "Stop" button in the user interface, before the boat has reached the endpoint & In the user interface the button "Running..." changes to "Calculate path". The boats turns off its motors and comes to a halt.\\ \hline
	\end{tabularx}
	\caption{Test of: Use case 13 - Stop coverage path - Main scenario}
\end{table}
