\section{Software unit tests}
\label{sec:unit_test}

Testing was not initially something both team members had long experience with, but it soon became clear that verification would become an important part of the development process, particularly software. 

One group member (Nicolai) had taken software testing the previous semester, and presented the general ideas and concepts before any controller code was written. 

The team decided that unit testing would be mandatory for all C++ code that was possible to test, since it offered great benefits, and potentially a whole new way of thinking about software development. This also means that all classes that requires testing need an interface, to be able to make mock and fake classes.

After debating and attempting different testing frameworks, it was decided that the Boost unit testing framework and the FakeIt isolation framework would be used. 

This combination proved to be very effective, but since both members of the team had no experience with either framework, there were a few issues, and a lot of learning, before the frameworks were used to their full potential.

At the end of development, over 170 tests totaling more than 8000 lines of code were written. 

Listing ~\ref{lst:servo_test} shows an example of a unit test of the Servo class.

\begin{lstlisting}[caption = {Servo test of SetPosition}, captionpos=b, label={lst:servo_test}, language=C++,firstnumber=1]
BOOST_AUTO_TEST_CASE(Servo_test_SetPosition_50_returns_1500)
{
	fakeit::Mock<IGPIO> gpioMock;
	fakeit::Fake(Method(gpioMock, GpioServo));
	fakeit::Fake(Method(gpioMock, GpioSetMode));

	IGPIO & gpio = gpioMock.get();

	Servo uut(gpio);

	uut.SetPosition(50);

	fakeit::Verify(Method(gpioMock, GpioServo).Matching([](auto gpio, auto pulsewidth)
	{
		//Ignore the gpio, since it could change and is not important
		return pulsewidth == 1500;
	})).Exactly(2); //It should call it once in the constructor and once again when we set it.
}
\end{lstlisting}

Unit testing has three phases, Arrange, Act, and Assert. 

In the example, mocks and fakes are first set up, or Arranged, to isolate the test from its external dependencies; other classes and functions. This way, an error in a function will only lead to errors when testing that function, and not when testing other functions which depend on it. 

The unit under test (uut) is then constructed, and a function is called; this is the Act step. 

Finally, the Assert step verifies that the function did exactly what was intended, in this case calling another function with a specific argument matching a specific value. This is done using a lambda function.

Most unit tests in the project are much more complicated than this simple example, but the general structure and procedure is the same. 

To have something with which to compare the navigation algorithm outputs in tests, two online example geographic calculators were sometimes used.\cite{movable-type}\cite{javawa}, the latter of which makes use of the Geographic library also.